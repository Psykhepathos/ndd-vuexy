\documentclass[12pt,a4paper]{article}
\usepackage[utf8]{inputenc}
\usepackage[brazilian]{babel}
\usepackage[margin=2.5cm]{geometry}
\usepackage{graphicx}
\usepackage{fancyhdr}
\usepackage{titlesec}
\usepackage{enumitem}
\usepackage{xcolor}
\usepackage{hyperref}
\usepackage{tabularx}
\usepackage{booktabs}
\usepackage{float}
\usepackage{subcaption}

% Configuração de cores
\definecolor{nddblue}{RGB}{0,82,147}
\definecolor{nddgray}{RGB}{128,128,128}

% Configuração de cabeçalho e rodapé
\pagestyle{fancy}
\fancyhf{}
\fancyhead[L]{\textcolor{nddblue}{\textbf{Sistema NDD Transport}}}
\fancyhead[R]{\textcolor{nddgray}{\today}}
\fancyfoot[C]{\textcolor{nddgray}{\thepage}}
\renewcommand{\headrulewidth}{2pt}
\renewcommand{\headrule}{\hbox to\headwidth{\color{nddblue}\leaders\hrule height \headrulewidth\hfill}}

% Configuração de títulos
\titleformat{\section}
{\color{nddblue}\Large\bfseries}
{\thesection}{1em}{}
\titleformat{\subsection}
{\color{nddblue}\large\bfseries}
{\thesubsection}{1em}{}

% Configuração de hyperlinks
\hypersetup{
    colorlinks=true,
    linkcolor=nddblue,
    filecolor=nddblue,
    urlcolor=nddblue,
}

\begin{document}

% Página de título
\begin{titlepage}
    \centering
    \vspace*{2cm}

    {\Huge\textcolor{nddblue}{\textbf{Sistema NDD Transport}}}\\[0.5cm]
    {\LARGE\textcolor{nddgray}{Sistema Integrado de Gestão de Transporte}}\\[1.5cm]

    {\Large\textbf{Plataforma Completa de Operações}}\\[0.3cm]
    {\large Integrando APIs NDD + Sistema Corporativo}\\[2cm]

    % Logo sem quadrado e sem caption
    \begin{figure}[H]
        \centering
        \includegraphics[width=0.5\linewidth]{Logo Tambasa.png}
    \end{figure}


    \vfill

    {\large Versão: 1.0}\\[0.3cm]
    {\large Data: \today}\\[0.3cm]
    {\large Desenvolvido por: Equipe NDD}

\end{titlepage}

% Sumário
\tableofcontents
\newpage

\section{Visão Geral do Projeto}

\subsection{O que é o Sistema NDD Transport}
O Sistema NDD Transport é uma plataforma moderna e completa para gestão de transporte que centraliza todas as operações necessárias para o dia a dia da Tambasa. O sistema conecta diretamente com as APIs oficiais da NDD e integra funcionalidades da integração já feita do SemParar, oferecendo uma solução única para todas as necessidades de operação Vale Pedágio e CIOT.

\subsection{Objetivos do Sistema}
\begin{itemize}[leftmargin=2cm]
    \item \textbf{Simplicidade:} Interface fácil de usar, mesmo para usuários não técnicos
    \item \textbf{Integração Total:} Conecta com sistemas NDD e banco corporativo
    \item \textbf{Economia de Tempo:} Automatiza processos que hoje são manuais
    \item \textbf{Conformidade Legal:} Garante que todos os documentos estejam de acordo com a legislação
    \item \textbf{Controle Total:} Visão completa de todas as operações em um só lugar
\end{itemize}

\section{Base Técnica do Sistema}

\subsection{Tecnologias Utilizadas}
\begin{table}[H]
    \centering
    \begin{tabularx}{\textwidth}{|X|X|}
        \hline
        \textbf{Componente} & \textbf{Tecnologia} \\
        \hline
        Interface do Usuário & Vue.js + Material Design (Vuexy) \\
        Sistema Backend & Laravel (PHP) \\
        Banco de Dados & Progress OpenEdge (sistema corporativo) \\
        Integrações & APIs NDD \\
        Segurança & Autenticação integrada \\
        \hline
    \end{tabularx}
    \caption{Tecnologias que fazem o sistema funcionar}
\end{table}

\subsection{Como o Sistema Funciona}
\begin{figure}[H]
    \centering
    \fbox{\begin{minipage}{14cm}
        \centering
        \vspace{1cm}
        {\large \textbf{FLUXO DO SISTEMA}}\\[0.5cm]
        Usuário → Interface Web → Servidor → Banco Corporativo\\[0.5cm]
        {\footnotesize ↕ Integração Automática ↕}\\[0.3cm]
        {\large \textbf{APIs NDD + SemParar}}\\[0.3cm]
        {\footnotesize (CIOT, Vale Pedágio, Pagamentos, Rotas)}
        \vspace{1cm}
    \end{minipage}}
    \caption{Como o sistema conecta tudo automaticamente}
\end{figure}

\section{O que Já Está Funcionando}

\textbf{Demonstração Atual:} O sistema já possui uma versão funcional que demonstra a interface moderna e as conexões com o sistema corporativo Progress. Esta versão serve para mostrar como ficará o produto final.

\subsection{Acesso Seguro ao Sistema}
\begin{itemize}
    \item Tela de login moderna e segura
    \item Controle de permissões por usuário
    \item Sessões protegidas
\end{itemize}

% Screenshot sem quadrado e sem caption
\begin{figure}[H]
    \centering
    \vspace{1cm} % espaço acima, se quiser
    \includegraphics[width=1\linewidth]{login.png}
    \vspace{1cm} % espaço abaixo, se quiser
    \caption{Tela de Login do sistema}
\end{figure}


\subsection{Painel Principal}
\begin{itemize}
    \item Visão geral de todas as operações do dia
    \item Números importantes em tempo real
    \item Navegação fácil para todas as funções
\end{itemize}

% Screenshot sem quadrado e sem caption
\begin{figure}[H]
    \centering
    \vspace{1cm} % espaço acima, se quiser
    \includegraphics[width=1\linewidth]{dashboard.png}
    \vspace{1cm} % espaço abaixo, se quiser
    \caption{Painel principal com resumo das operações}
\end{figure}

\subsection{Gestão de Transportadores}
\begin{itemize}
    \item Lista completa com mais de 6.900 transportadores
    \item Busca rápida por nome ou código
    \item Interface limpa e fácil de usar
    \item Dados sempre atualizados do sistema corporativo
\end{itemize}

% Screenshot sem quadrado e sem caption
\begin{figure}[H]
    \centering
    \vspace{1cm} % espaço acima, se quiser
    \includegraphics[width=1\linewidth]{transportadores.png}
    \vspace{1cm} % espaço abaixo, se quiser
    \caption{Tela de consulta de transportadores e motoristas}
\end{figure}

\subsection{Gerenciamento de Pacotes}
\begin{itemize}
    \item Lista completa de todos os pacotes carregados do sistema corporativo
    \item Filtros avançados por data, rota, status e transportador
    \item Busca rápida por código do pacote ou cliente
    \item Visualização detalhada de cargas e entregas por pacote
    \item Identificação automática de pacotes TCD para transferência entre CDs
\end{itemize}

% Placeholder para screenshot
\begin{figure}[H]
    \centering
    \vspace{1cm} % espaço acima, se quiser
    \includegraphics[width=1\linewidth]{pacote.png}
    \vspace{1cm} % espaço abaixo, se quiser
    \caption{Tela de gerenciamento completo de pacotes}
\end{figure}

\subsection{Vista de Itinerário de Entrega}
\begin{itemize}
    \item Visualização detalhada de cada ponto de entrega do pacote
    \item Mapa interativo mostrando a sequência de entregas
    \item Informações completas de cada cliente (endereço, GPS, observações)
    \item Otimização automática da ordem das entregas
    \item Exportação para impressão ou sistema de bordo
\end{itemize}

% Placeholder para screenshot
\begin{figure}[H]
    \centering
    \vspace{1cm} % espaço acima, se quiser
    \includegraphics[width=1\linewidth]{itinerario.png}
    \vspace{1cm} % espaço abaixo, se quiser
    \caption{Vista detalhada do itinerário de entregas por pacote}
\end{figure}

\subsection{Calculadora de Vale Pedágio (Estilo WebRouter)}
\begin{itemize}
    \item Arrastar e soltar pontos no mapa para reorganizar
    \item Mapas reais do Google para calcular rotas
    \item Cálculo automático de valores de pedágio
    \item Funciona no computador, tablet e celular
\end{itemize}

% Placeholder para screenshot
\begin{figure}[H]
    \centering
    \vspace{1cm} % espaço acima, se quiser
    \includegraphics[width=1\linewidth]{calculadora.png}
    \vspace{1cm} % espaço abaixo, se quiser
    \caption{Calculadora de vale pedágio com mapa interativo}
\end{figure}

\subsection{Planejamento de Rotas}
\begin{itemize}
    \item Carrega automaticamente os dados das entregas
    \item Mostra até 15+ pontos de entrega por viagem
    \item Organiza em páginas para facilitar a visualização
    \item Cores diferentes para início, meio e fim da rota
\end{itemize}

% Placeholder para screenshot
\begin{figure}[H]
    \centering
    \vspace{1cm} % espaço acima, se quiser
    \includegraphics[width=1\linewidth]{planejamento.png}
    \vspace{1cm} % espaço abaixo, se quiser
    \caption{Sistema de planejamento de rotas de entrega}
\end{figure}

\section{Funcionalidades que Serão Implementadas}

O sistema será integrado com as APIs oficiais da NDD e incluirá todas as funcionalidades que já foram implementadas do SemParar no corporativo. Abaixo estão as principais funções que você poderá usar:

\subsection{Gestão de Vale Pedágio}
\textbf{O que faz:} Cria e gerencia vales pedágio eletrônicos automaticamente
\begin{itemize}
    \item \textbf{Criação Automática:} Sistema cria o vale baseado na rota planejada
    \item \textbf{Dados Completos:} Inclui informações do motorista, veículo e rota
    \item \textbf{Valores Reais:} Calcula valores de pedágio atualizados
    \item \textbf{Impressão:} Gera documentos para impressão quando necessário
    \item \textbf{Controle:} Acompanha status dos vales emitidos
\end{itemize}

\subsection{Sistema CIOT (Conhecimento de Transporte)}
\textbf{O que faz:} Gera documentos oficiais obrigatórios para transporte de carga
\begin{itemize}
    \item \textbf{Emissão Automática:} Cria CIOT baseado nos dados da carga
    \item \textbf{Conformidade Legal:} Garante que está de acordo com a ANTT
    \item \textbf{Dados da Carga:} Inclui remetente, destinatário e detalhes da mercadoria
    \item \textbf{Cancelamento:} Permite cancelar CIOTs quando necessário
    \item \textbf{Histórico:} Mantém registro de todos os documentos emitidos
\end{itemize}

\subsection{Sistema TCD }
\textbf{O que faz:} Identifica e gerencia automaticamente transferências entre centros de distribuição
\begin{itemize}
    \item \textbf{Detecção Automática:} Sistema identifica pacotes de transferência CD
    \item \textbf{Classificação TCD:} Marca automaticamente como TCD
    \item \textbf{Documentação Específica:} Gera fluxo adequado para transporte e pedidos
    \item \textbf{Controle de Origem/Destino:} Rastreia movimentação entre centros de distribuição
    \item \textbf{Integração ANTT:} Cumpre regulamentações específicas para TCD
    \item \textbf{Relatórios Separados:} Separa operações TCD das demais atividades
\end{itemize}

\subsection{Otimização de Rotas}
\textbf{O que faz:} Calcula as melhores rotas para economizar tempo e combustível
\begin{itemize}
    \item \textbf{Rotas Inteligentes:} Encontra o caminho mais eficiente
    \item \textbf{Cálculo de Distância:} Mostra quilometragem real
    \item \textbf{Tempo de Viagem:} Estima duração da viagem
    \item \textbf{Custos:} Calcula combustível e pedágio (Ainda vendo se é funcional)
    \item \textbf{Alternativas:} Oferece opções de rota diferentes
\end{itemize}

\subsection{Sistema de Pagamentos}
\textbf{O que faz:} Processa pagamentos de pedágio e outras despesas automaticamente
\begin{itemize}
    \item \textbf{Pagamento Imediato:} Efetua pagamentos na hora
    \item \textbf{Verificação:} Confirma se pagamentos foram processados
    \item \textbf{Histórico:} Mantém registro de todas as transações
    \item \textbf{Relatórios:} Gera resumos de gastos por período
    \item \textbf{Controle:} Permite acompanhar status dos pagamentos
\end{itemize}

\subsection{Consulta de Transportadores}
\textbf{O que faz:} Busca informações detalhadas de empresas transportadoras
\begin{itemize}
    \item \textbf{Dados Completos:} RNTRC, razão social, endereço
    \item \textbf{Situação Cadastral:} Verifica se transportador está regular
    \item \textbf{Histórico:} Consulta operações anteriores
    \item \textbf{Validação:} Confirma dados antes de contratar
    \item \textbf{Relatórios:} Gera listas de transportadores
\end{itemize}

\subsection{Geração de Documentos}
\textbf{O que faz:} Cria automaticamente todos os documentos necessários
\begin{itemize}
    \item \textbf{CIOT:} Conhecimento de transporte oficial
    \item \textbf{Vale Pedágio:} Comprovantes de pagamento
    \item \textbf{Relatórios:} Resumos de operações
    \item \textbf{PDF:} Documentos prontos para impressão
    \item \textbf{Backup:} Mantém cópias digitais seguras
\end{itemize}

\subsection{Funcionalidades do SemParar Integradas}
\textbf{O que será incluído:} Todas as funções que você já conhece do SemParar
\begin{itemize}
    \item \textbf{Cálculo de Pedagio:} Como no site do SemParar e integração atual do Corporativo
    \item \textbf{Múltiplas Rotas:} Comparação de caminhos alternativos (NOVO)
    \item \textbf{Mapas Interativos:} Visualização das rotas no mapa
    \item \textbf{Custos Detalhados:} Breakdown completo dos valores
    \item \textbf{Histórico de Rotas:} Salva rotas já calculadas
    \item \textbf{Exportação:} Salva resultados em Excel/PDF
\end{itemize}

\section{Etapas do Projeto}

\subsection{Primeira Etapa - MVP (Concluída)}
\textbf{Objetivo:} Criar o Produto Mínimo Viável para apresentar a tecnologia e validar a solução
\textbf{Finalidade:} Demonstrar telas, interface, tecnologia e uso do sistema para obter aprovação

Esta etapa serve como \textbf{prova de conceito} para mostrar:
\begin{itemize}[leftmargin=3cm]
    \item[\checkmark] Interface moderna e profissional criada
    \item[\checkmark] Conexão com sistema corporativo Progress funcionando
    \item[\checkmark] Sistema de login seguro e controle de acesso
    \item[\checkmark] Cadastro de transportadores com dados reais
    \item[\checkmark] Gerenciamento de pacotes com identificação TCD
    \item[\checkmark] Vista detalhada de itinerários de entrega
    \item[\checkmark] Calculadora de vale pedágio básica funcional
    \item[\checkmark] Visualização de rotas no mapa interativo
    \item[\checkmark] Sistema responsivo (computador, tablet e celular)
\end{itemize}

\textbf{Status:} \textcolor{green}{\textbf{CONCLUÍDA}} - Pronta para apresentação e validação

\subsection{Segunda Etapa - Implementação das APIs (Próxima)}
\textbf{Objetivo:} Desenvolver e implementar cada uma das 15+ APIs NDD com testes completos


\subsubsection{Integração com APIs NDD}
\begin{itemize}
    \item Conexão com todas as 15+ APIs da NDD
    \item Sistema completo de CIOT (criar, consultar, cancelar)
    \item Vale Pedágio automático completo
    \item Sistema de pagamentos integrado
    \item Geração automática de todos os documentos
\end{itemize}

\subsubsection{Funcionalidades do SemParar}
\begin{itemize}
    \item Todas as funções de cálculo de pedágio
    \item Comparação de rotas alternativas
    \item Mapas interativos completos
    \item Histórico de consultas
    \item Exportação para Excel e PDF
    \item Sistema de favoritos
\end{itemize}

\textbf{Deliverables da Segunda Etapa:}
\begin{itemize}
    \item Sistema CIOT completo (criar, consultar, cancelar)
    \item Vale Pedágio automático integrado com NDD
    \item Sistema de pagamentos funcionando
    \item Todas as funcionalidades do WebRouter
    \item Documentação técnica completa
    \item Testes de integração aprovados
\end{itemize}

\subsection{Terceira Etapa - Refinamento e Testes em Produção}
\textbf{Objetivo:} Aperfeiçoar o sistema e realizar testes intensivos no ambiente real


\textbf{Atividades da Terceira Etapa:}
\begin{itemize}
    \item \textbf{Testes em Produção:} Validação com usuários reais da Tambasa
    \item \textbf{Ajustes de Performance:} Otimização de velocidade e estabilidade
    \item \textbf{Correção de Bugs:} Resolução de problemas encontrados no uso
    \item \textbf{Refinamento da Interface:} Melhorias baseadas no feedback dos usuários
    \item \textbf{Documentação Final:} Manuais completos e treinamento
    \item \textbf{Backup e Segurança:} Implementação de rotinas de backup
\end{itemize}

\textbf{Ao final da Terceira Etapa:} \textcolor{green}{\textbf{IMPLEMENTAÇÃO COMPLETA}}

\subsection{Projetos Futuros - Melhorias Contínuas}
\textbf{Objetivo:} Evoluir o sistema além do escopo atual para aumentar qualidade e facilidade de trabalho

Após a conclusão da implementação, iniciaremos um \textbf{novo ciclo de projetos} focado em:
\begin{itemize}
    \item \textbf{Inteligência Artificial:} Sugestões automáticas de rotas
    \item \textbf{Relatórios Avançados:} Business Intelligence e Analytics
    \item \textbf{Integração Mobile:} App nativo para motoristas
    \item \textbf{IoT Integration:} Sensores de veículos e cargas
    \item \textbf{Automação Total:} Fluxos de trabalho completamente automatizados
    \item \textbf{Expansão de Funcionalidades:} Recursos solicitados pelos usuários
\end{itemize}

\section{Requisitos Técnicos}

\subsection{O que é Necessário para o Sistema Funcionar}
\begin{table}[H]
    \centering
    \begin{tabularx}{\textwidth}{|l|X|}
        \hline
        \textbf{Item} & \textbf{Necessário} \\
        \hline
        Servidor & Computador com 16GB de memória \\
        Espaço em Disco & 50GB livres \\
        Internet & Conexão estável para APIs \\
        Banco Progress & Versão atual do sistema corporativo \\
        Certificado Digital & A3 ou A1 para documentos oficiais \\
        \hline
    \end{tabularx}
    \caption{Requisitos para funcionamento do sistema}
\end{table}

\subsection{Como Acessar o Sistema}
\begin{itemize}
    \item \textbf{Demonstração:} Sistema funcionando para mostrar as telas
    \item \textbf{Produção:} Sistema final instalado no servidor da empresa
    \item \textbf{Acesso:} Qualquer computador, tablet ou celular com internet
    \item \textbf{Segurança:} Conexões protegidas e dados criptografados
    \item \textbf{Login de Teste:} admin@ndd.com / 123456
\end{itemize}

\section{O que Você Pode Cobrar na Primeira Etapa}

\subsection{Funcionalidades que Devem Estar Funcionando}
\begin{enumerate}
    \item \textbf{Sistema no ar} - Você consegue acessar de qualquer lugar
    \item \textbf{Telas bonitas} - Interface moderna e fácil de usar
    \item \textbf{Dados reais} - Informações do sistema corporativo aparecem
    \item \textbf{Cadastro de transportadores} - Lista, busca e visualiza
    \item \textbf{Gerenciamento de pacotes} - Lista completa com identificação TCD
    \item \textbf{Vista de itinerários} - Detalhamento completo das entregas
    \item \textbf{Vale pedágio básico} - Calcula valores e rotas
    \item \textbf{Login seguro} - Acesso protegido ao sistema
\end{enumerate}

\subsection{Documentação que Deve Ser Entregue}
\begin{itemize}
    \item Manual de como usar o sistema
    \item Lista completa das 15+ APIs que serão implementadas
    \item Instruções de instalação
    \item Plano detalhado da segunda etapa
\end{itemize}

\section{Como Saber se Está Funcionando Bem}

\subsection{Checklist de Funcionamento}
\begin{table}[H]
    \centering
    \begin{tabularx}{\textwidth}{|X|c|}
        \hline
        \textbf{O que Testar} & \textbf{Deve Funcionar} \\
        \hline
        As páginas abrem rapidamente (menos de 2 segundos) & ✓ \\
        O sistema não trava nem sai do ar & ✓ \\
        Completa integração com Corporativo & ✓ \\
        Funciona bem no celular e tablet & ✓ \\
        Os dados do Progress aparecem corretamente & ✓ \\
        \hline
    \end{tabularx}
    \caption{Lista de verificação para aceitar a primeira etapa}
\end{table}

\subsection{O que Você Deve Ver Funcionando}
\begin{itemize}
    \item \textbf{Visual Profissional:} Telas limpas e organizadas
    \item \textbf{Dados Corretos:} Informações da Tambasa aparecem certinhas
    \item \textbf{Velocidade:} Sistema responde rápido, sem travamentos
    \item \textbf{Facilidade:} Qualquer pessoa consegue usar sem dificuldade
    \item \textbf{Base Sólida:} Tudo preparado para crescer na segunda etapa
\end{itemize}

\section{Próximos Passos}

\subsection{O que Fazer Após a Primeira Etapa}
\begin{enumerate}
    \item Coletar sugestões de melhorias
    \item Documentar o que precisa ser ajustado
\end{enumerate}

\subsection{Preparação da Segunda Etapa}
\begin{enumerate}
    \item Planejar implementação das APIs NDD
\end{enumerate}

\vfill

\section*{Conclusão e Solicitação de Aval}

\subsection*{Visão Geral do Projeto}

O \textbf{Sistema NDD Transport} representa uma solução moderna e abrangente para revolucionar as operações de transporte da Tambasa. O projeto foi estruturado em fases bem definidas, cada uma com objetivos claros e entregas específicas.

\subsection*{Primeira Etapa - MVP Concluída com Sucesso}

A \textcolor{green}{\textbf{Primeira apresentação MVP foi concluída}} e está \textbf{funcionando perfeitamente}. Esta fase teve como objetivo apresentar a tecnologia, demonstrar as telas, validar o uso e verificar se a solução está alinhada com as necessidades da empresa.

\textbf{Resultados Alcançados:}
\begin{itemize}
    \item Interface moderna conectada ao sistema corporativo Progress
    \item Demonstração funcional de todas as telas principais
    \item Validação da tecnologia e arquitetura escolhidas
    \item Prova de conceito aprovada e funcionando
\end{itemize}

\subsection*{Roadmap das Próximas Etapas}

\textbf{Segunda Etapa - Implementação das APIs:}
\begin{itemize}
    \item Desenvolvimento de cada uma das 15+ APIs da NDD
    \item Implementação completa do sistema CIOT
    \item Integração total com Vale Pedágio automático
    \item Testes extensivos de cada funcionalidade
    \item Todas as funcionalidades do WebRouter implementadas
\end{itemize}

\textbf{Terceira Etapa - Refinamento e Produção:}
\begin{itemize}
    \item Testes intensivos no ambiente real da Tambasa
    \item Ajustes baseados no feedback dos usuários
    \item Otimização de performance e estabilidade
    \item Documentação final e treinamento completo
    \item \textcolor{green}{\textbf{IMPLEMENTAÇÃO COMPLETAMENTE FINALIZADA}}
\end{itemize}

\textbf{Projetos Futuros - Melhorias Contínuas:}
Após a implementação completa, iniciaremos um novo ciclo de projetos para melhorar ainda mais a qualidade e facilidade do trabalho, incluindo inteligência artificial, analytics avançados e automação total.

\subsection*{Objetivos Claros do Trabalho}

\textbf{O que será entregue:}
\begin{enumerate}
    \item Sistema completo funcionando com todas as APIs NDD integradas
    \item Interface moderna e fácil de usar para todos os usuários
    \item Integração total com sistema corporativo Progress
    \item Automação de processos de CIOT e Vale Pedágio
    \item Documentação completa e treinamento dos usuários
    \item Sistema testado e aprovado para uso em produção
\end{enumerate}

\textbf{Benefícios para a Tambasa:}
\begin{itemize}
    \item Redução drástica do tempo gasto em processos manuais
    \item Conformidade automática com todas as regulamentações
    \item Controle total e visibilidade de todas as operações
    \item Economia significativa de custos operacionais
    \item Modernização tecnológica da área de transporte
\end{itemize}

\subsection*{Solicitação de Aval para Desenvolvimento}

Com base na \textbf{demonstração bem-sucedida do MVP} e tendo apresentado claramente:
\begin{itemize}
    \item Os objetivos e escopo do projeto
    \item O cronograma detalhado das etapas
    \item Os resultados esperados e benefícios
    \item A tecnologia validada e funcionando
\end{itemize}

\textbf{Solicitamos formalmente o aval para dar início ao desenvolvimento da Segunda Etapa}, que implementará todas as funcionalidades das APIs NDD e transformará o sistema em uma ferramenta completa para uso diário nas operações de transporte da Tambasa.

\textbf{Confirmação de Entendimento:}
Estamos cientes de que o objetivo é criar um sistema completo, integrado e funcional que automatize e modernize todos os processos de transporte, proporcionando economia de tempo, redução de custos e total conformidade legal.

\textcolor{nddblue}{\textbf{AGUARDAMOS SUA APROVAÇÃO PARA INICIAR A IMPLEMENTAÇÃO COMPLETA.}}

\textbf{Status Atual:} \textcolor{green}{\textbf{MVP Pronto e Aprovado}} \\
\textbf{Próximo Passo:} \textcolor{blue}{\textbf{Aguardando Aval para Segunda Etapa}}

\end{document}
